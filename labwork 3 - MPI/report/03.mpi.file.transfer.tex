\documentclass{article}


\title{Report #3: MPI}
\author{Le Trung Nghia - USTHBI5-098 ** Ta Nguyen Long - USTHBI5-080 ** Do Huy Manh - USTHBI6-093 ** Nguyen Trong Nhan - USTHBI4-115 }
\date{February 2018}

\usepackage{graphicx}
\usepackage{amsfonts}
\usepackage{amssymb}



\begin{document}
1. Desgin the MPI service

Complier: rpcgen –C -a

2. Organize system
 - Figure is attached
 
3. Implement the file transfer
- int dest = 1
  int source = 0
—> destination, which rank to send

- if (world_rank == 0 ){
printf("This is Send\n");
—> send the file

- int c,temp;
temp=0;
char fileName[100];
printf("Enter name of the file to send :\n");
scanf("%s", fileName);
f = fopen(fileName,"r");
—> read file

MPI_Send(&lengthContent, count, MPI_INT, dest, tagContentSize, MPI_COMM_WORLD);
tem = sizeof(lengthContent)/sizeof(int);
printf("this is temp for lengthContent%d\n", tem);

MPI_Send(fileName, sizeof(fileName)/sizeof(char), MPI_CHAR, dest, tagName, MPI_COMM_WORLD);
tem = sizeof(fileName)/sizeof(char);
printf("this is temp for filename%d\n", tem);

MPI_Send(content, lengthContent, MPI_CHAR, dest, tagContent, MPI_COMM_WORLD);
printf("Send successfuly\n");
—> Send using MPI

4. Who does what!
Le Trung Nghia: work on RPC section
Ta Nguyen Long + Do Huy Manh: work on MPI section
Do Huy Manh + Nguyen Trong Nhan: work on the report

\end{document}
